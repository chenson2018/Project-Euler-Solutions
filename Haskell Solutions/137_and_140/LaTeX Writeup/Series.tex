\documentclass[12pt]{article}
\usepackage{amsmath}
\usepackage{amssymb}
\usepackage[svgnames]{xcolor}
\usepackage[colorlinks=true, linkcolor=Maroon, urlcolor=Blue]{hyperref}
\setlength{\parskip}{.75em}

\begin{document}

\noindent{Chris Henson}\\
Project Euler Questions 137/140\\

This short write-up will walk through some of the mathematical derivations that I used to solve Project Euler questions 137/140.

First, let us recall the definition of the Fibbonacci numbers. We start with:

$$
F_0=0,\quad F_1= 1
$$

and define further numbers by the relation:

$$
F_n=F_{n-1} + F_{n-2}
$$

This gives us the first few terms:

$$
0,\;1,\;1,\;2,\;3,\;5,\;8,\;13,\;21,\;34,\;55,\;89,\;144,\; \ldots
$$

(Please note that Project Euler omits the first index, but this will not change our calculations in any meaningful way, and I will try to be as explicit as possible about this.)

The natural generalization is to consider a different starting point. I will write 

$$
G(a, b, n) = G(a, b, n-1) + G(a, b, n-2)
$$

given the starting point:

$$
G(a, b, 0) = a,\quad G(a, b, 1) = b
$$

and when we are working with a set $a$ and $b$ I will simply write $G_n$. You can see that $G(0, 1 , n)$ gives the Fibbonacci numbers themselves.

\newpage

These two Project Euler questions are concerned with the power series with coefficients given by the above sequences. That is, we are interested in the value of the summations:

$$
\sum_{k=1}^\infty G(a, b, n) x^k = G(a, b, 1)x + G(a, b, 2)x^2 + G(a, b, 3)x^3 + \dots
$$

Again, if we take $a=0$ and $b=1$, we will have the Fibbonacci numbers:

\begin{align*}
\sum_{k=1}^\infty G(0, 1, n) x^k &= F_1x + F_2x^2 + F_3x^3 + \dots \\
					  & = x + x^2 + 2x^3 + 3x^4 +5x^5 \dots
\end{align*}

Interestingly, it is not too difficult to derive a closed form solution to these series. Let us fix $G_0 = a$ and $G_1 = b$. Then the power series is:

\begin{align*}
\sum_{k=0}^\infty G_k x^k = G_0 + G_1 x + G_2 x^2 + \dots
\end{align*}

Please note that I have included the initial term $G_0$ that Project Euler omits. We can correct for this at the end by simply subtracting this term.

\newpage

First, I will pull out the first two given terms from the series:

\begin{align*}
\sum_{k=0}^\infty G_k x^k = G_0 + G_1x + \sum_{k=2}^\infty G_k x^k
\end{align*}

Now I can Replace $G_k$  with the relation that defines it, and split apart the summation further:


\begin{align*}
\sum_{k=0}^\infty G_k x^k &= G_0 + G_1x + \sum_{k=2}^\infty (G_{k-1} + G_{k-2}) x^k \\
				   &= G_0 + G_1x +  \sum_{k=2}^\infty G_{k-1} x^k + \sum_{k=2}^\infty G_{k-2} x^k
\end{align*}

Next, I notice that multiplying my original series by $x$ and $x^2$ gives:

\begin{align*}
x\sum_{k=0}^\infty G_k x^k &= G_0x + G_1 x^2 + G_2 x^3 + \dots\\
x^2\sum_{k=0}^\infty G_k x^k &= G_0x^2 + G_1 x^3 + G_2 x^4 + \dots
\end{align*}

Substituting these into the above equation I have:

\begin{align*}
\sum_{k=0}^\infty G_k x^k &= G_0 + G_1 x + x\sum_{k=0}^\infty G_k x^k - G_0 x + x^2\sum_{k=0}^\infty G_k x^k
\end{align*}

\newpage

Denoting the original summation as $s(x)$ this is:

$$
s(x) =  G_0 + G_1 x + xs(x) - G_0 x + x^2 s(x)
$$ 

and solving gives:

$$
s(x) = \dfrac{G_0 - G_0 x + G_1 x}{1 - x - x^2}
$$

Remember that this includes our first term. So we have\footnote{

Note Project Euler omits the first term, considering instead the series:

$$
\sum_{k=1}^\infty G_k x^k = G_1 x + G_2 x^2 + \dots =  \dfrac{G_0 - G_0 x + G_1 x}{1 - x - x^2} - G_0
$$

}:

$$
\sum_{k=0}^\infty G_k x^k = G_0 + G_1 x + G_2 x^2 + \dots =  \dfrac{G_0 - G_0 x + G_1 x}{1 - x - x^2}
$$

For the Fibbonacci series we set $G_0 = 0$ and $G_1 = 1$ giving:

$$
\sum_{k=0}^\infty F_k x^k = F_0 + F_1 x + F_2 x^2 + \dots =  \dfrac{x}{1 - x - x^2}
$$


\newpage

The next question under consideration is which rational values of x give an integer value for our power series? I will write:

$$
s(x) = \sum_{k=1}^\infty G_k x^k = G_0 + G_1 x + G_2 x^2
$$

Above we had the equation:

$$
s(x) =  G_0 + G_1 x + xs(x) - G_0 x + x^2 s(x)
$$ 

which solving for x using the quadratic formula\footnote{We actually have two choices for roots, but one is outside the radius of convergence for the power series} gives us:

$$
x = \dfrac{-(G_1 - G_0 + s) + \sqrt{(G_1 - G_0 + s)^2 - 4s(G_0-s)}}{2s}
$$

Remember that we included the term $G_0$. If we would like to correct for this, we can make the transformation $s \rightarrow S - G_0$ where:

\begin{align*}
S(x)&= \sum_{k=1}^\infty G_k x^k = \\
       &= \sum_{k=0}^\infty G_k x^k - G_0\\
       &= s(x) - G_0
\end{align*}

which gives instead:

$$
x = \dfrac{-(G_1 + S) + \sqrt{(G_1 + S)^2 - 4(S + G_0)(-S)}}{2(S + G_0)}
$$

\newpage

Using the second equation and taking $G_0 = 0$ and $G_1 = 1$ to give the Fibbonacci power series, we have have the following values:

\begin{align*}
S\left(\sqrt{2} -1\right) &= 1\\
S\left(\dfrac{1}{2}\right) &= 2\\
S\left(\dfrac{\sqrt{13} - 2}{3}\right) &= 3\\
S\left(\dfrac{\sqrt{89} - 5}{8}\right) &= 4\\
\vdots
\end{align*}

We can see for instance that $S = 2$ has the desired property that $x$ is rational. What conditions must be met for this? First, look at our equation for x we we substitue the particuliar values for $G_0$ and $G_1$:

$$
x = \dfrac{-(1 + S) + \sqrt{(1 + S)^2 - (2S)^2}}{2S}
$$

This will be rational exactly when the discriminant $(1 + S)^2 - (2S)^2$ is a perfect square \footnote{The following arguement follows from this article : \href{https://github.com/chenson2018/Project-Euler-Solutions/blob/master/Haskell\%20Solutions/137_and_140/Reference.pdf/}{Link}} 

\newpage

Since the discriminant is itself a sum of two squares, we can consider them as Pythagorean triples so that for integers m, n:

$$
S + 1 = m^2 - n^2,\quad 2S = 2mn
$$

This gives the equation:

$$
m^2 - n^2 = mn + 1
$$

which can be written as

$$
(2m-n)^2 = 4 + 5n^2
$$

We need $4 + 5n^2$ to be a perfect square to ensure a rational x. Unexpectedly, using $n = F_{2j}$ (where j is a positive integer) ensures this is satisfied:

\begin{align*}
4 + 5n^2 &= 4 + 5F_{2j} = 4 + 5 \left( \frac{(\varphi^{2j}-\psi^{2j})^2}{5} \right)\\
                &= 4 + \varphi^{4j} - 2(\varphi\psi)^{2j} + \psi^{4j}\\
                &= \varphi^{4j} + 2 + \psi^{4j}\\
	     &=   \varphi^{4j} + 2 (\varphi\psi)^{2j}+ \psi^{4j}\\
	     &= (\varphi^{2j} + \psi^{2j})^2
\end{align*}



Furthermore, substituing into the equation $(2m-n)^2 = 4 + 5n^2$  above gives:

\begin{align*}
m &= \dfrac{1}{2} (\varphi^{2j} + \psi^{2j} + F_{2j})\\
    &= \dfrac{1}{2} \left( \varphi^{2j} + \psi^{2j} + \frac{\varphi^{2j}-\psi^{2j}}{\sqrt{5}} \right)\\
    &= \dfrac{\varphi\varphi^{2j} - \psi\psi^{2j}}{\sqrt5} = \dfrac{\varphi^{2j+1} - \psi^{2j+1}}{\sqrt5}\\
    &= F_{2j+1}
\end{align*}

\newpage

Finally, subsituting this back into the original series gives:

$$
\sum_{k=1}^\infty F_k (F_{2j}/F_{2j+1})^k = F_{2j}F_{2j+1}, \quad j \geq 1
$$

What an unexpected resutl! That we could consider successive coeffients in identify this rational points is certainly not intuitive. We see that these numbers also grow very quickly:

\begin{align*}
F_{2j}F_{2j+1}, n \in \mathbb{Z}^+= \{2, &15, 104, 714, 4895, 33552, 229970, 1576239, 10803704, 74049690,\\
					          &  507544127, 3478759200, 23843770274, 163427632719, 1120149658760, \dots\}
\end{align*}


The last number in this list is the solution to problem 137. Problem 140 considers the initalization $G_0 = 3$ and $G_1 = 1$, which give a more complicated sort of equation that determines the rational points. Still, they can be written entirely in terms of the Fibbonacci series.









\end{document}